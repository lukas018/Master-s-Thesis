\usepackage{blindtext} % This is just to get some nonsense text in this template, can be safely removed
\usepackage{csquotes} % Recommended by biblatex
\usepackage[linesnumbered,algoruled,boxed,lined]{algorithm2e}
\usepackage{bm}
\usepackage[urldate=long, maxbibnames=99]{biblatex}
\usepackage[hidelinks]{hyperref}
\usepackage[utf8]{inputenc}
\usepackage{lastpage}
\usepackage{amsmath}
\usepackage{amssymb}
\usepackage{microtype}
\usepackage{graphicx}
\usepackage{float}
\usepackage[acronym, smallcaps]{glossaries}
\usepackage[labelfont=bf]{caption}
\usepackage{tikz}
	\usetikzlibrary{positioning}
	\usepackage{subcaption}
\usepackage{caption}
\usepackage{catchfile}
\usepackage{pgfplots}% Provides options for creating plots
	\usepgfplotslibrary{groupplots}
\pgfplotsset{compat=1.15}
\usepackage{multirow}
\usepackage{pifont}
\usepackage{cleveref}
\usepackage{siunitx}

\sisetup{
  round-mode          = places, % Rounds numbers
  round-precision     = 2, % to 2 places
}

\newcommand{\cmark}{\ding{51}}%
\newcommand{\xmark}{\ding{55}}%


\usetikzlibrary{shapes, arrows}
\addbibresource{references.bib} % The file containing our references, in BibTeX format
%\captionsetup[subfigure]{labelformat=empty} % Remove letters on subcaption figures

\let\firstchar\lowercase
\let\oldprintglossaries\printglossaries
\def\printglossaries{\let\firstchar\uppercase\oldprintglossaries}

\loadglsentries{acronyms}
\makeglossaries

\setlength{\parindent}{0em}
\setlength{\parskip}{0.7em}

\newcommand{\No}{\mathcal{N}}
\newcommand{\bo}{\textbf}

%----------- Mathematic commands----------------
%------------------------------------
\newcommand{\argmax}[1]{\underset{#1}{\operatorname{argmax}}\;} %
\newcommand{\argmin}[1]{\underset{#1}{\operatorname{argmin}}\;} %
\newcommand{\pos}[1]{Pr \left( #1\right)}
\newcommand{\condpos}[2]{Pr \left( #1 \middle| #2 \right)}
\newcommand{\jointpos}[2]{Pr \left( #1, #2 \right)}
\newcommand{\innerprod}[2]{\langle #1 , #2 \rangle}
\newcommand{\partder}[2]{\frac{\partial#1}{\partial#2}}
\newcommand{\partdertext}[2]{\partial#1/\partial#2}
\newcommand{\fullder}[2]{\frac{d#1}{d#2}}
\newcommand{\maxfunc}[1]{\underset{#1}{\operatorname{max}}\;}
\newcommand{\gradient}[1]{\nabla#1}
\newcommand{\gradientsub}[2]{\nabla_{#2}#1}


\newcommand{\bacterium}[8]{
    \pgfmathsetmacro{\circleWidth}{#3*2 + 2.4}
    \pgfmathsetmacro{\circleCenterY}{#2 + 0.5*\circleWidth}
    
    \node[circle, below, minimum width= \circleWidth cm, minimum height= \circleWidth cm, label=#8:#5] (#5) at (#1, \circleCenterY) {};

    
    \pgfmathsetmacro{\randomizeA}{
        #1 + (rand*#3)
    }
    
    \pgfmathsetmacro{\randomizeB}{
        #2 + (rand*#3)
    }
    
    \pgfmathsetmacro{\randomizeImage}{
        int(rand*7)+ 8
    }
    
    \node[draw=black,very thick, inner sep=0pt] (\randomizeA) at (\randomizeA cm, \randomizeB cm) {
         \includegraphics[height=15pt, width=15pt]{#4/#7}
    };
    
    \node[rectangle, opacity=0.2, #6 inner sep=0pt, minimum width=15pt, minimum height=15pt] (#5_#7) at (\randomizeA cm, \randomizeB cm) {};
}


\newcommand{\randomCircle}[8]{
    \foreach \x in {1,...,#4}{
        \bacterium{#1}{#2}{#3}{#5}{#6}{#7}{\x}{#8}
    }
}

\tikzstyle{decision} = [diamond, draw, fill=blue!20, 
    text width=4.5em, text badly centered, node distance=3cm, inner sep=0pt]
\tikzstyle{block} = [rectangle, draw, fill=blue!20, 
    text width=5em, text centered, rounded corners, minimum height=4em]
\tikzstyle{line} = [draw, -latex']
\tikzstyle{cloud} = [draw, ellipse,fill=red!20, node distance=3cm, minimum height=2em]