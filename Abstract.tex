\begin{abstract}
Meta-Learning is an approach to machine learning that teaches models how to learn new tasks with only a handful of examples. However, meta-learning requires a large labeled dataset during its initial meta-learning phase, which restricts what domains meta-learning can be used in. This thesis investigates if this labeled dataset can be replaced with a synthetic dataset without a loss in performance. The approach has been tested on the task of military vehicle classification. The results show that for few-shot classification tasks, models trained with synthetic data can come close to the performance of models trained with real-world data. The results also show that adjustments to the data-generation process, such as light randomization, can have a significant effect on performance, suggesting that fine-tuning to the generation process could further increase performance.
\end{abstract}

\begin{otherlanguage}{swedish}
  \begin{sweabstract}
Metainlärning är en metod inom maskininlärning som gör det möjligt att lära en modell nya uppgifter med endast en handfull mängd träningsexempel. Metainlärning kräver dock en stor mängd träningsdata under själva metaträningsfasen, vilket begränsar vilka domäner som metodiken kan appliceras på. Detta examensarbete utreder om syntetisk bilddata, som genererats med hjälp av en simulator, kan ersätta verklig bilddata under metainlärning. Metoden har utvärderats på militär fordonsklassificering. Resultaten visar att en modell metainlärd med syntetisk data kan närma sig prestandan hos en modell metainlärd med riktig data för few-shot klassificering. Resultaten visar även att små ändringar i genereringsprocessen, exempelvis graden av slumpmässigt ljus, har en stor inverkan på den slutgiltiga prestandan, vilket ger hopp om att ytterligare finjustering av genereringsprocessen kan resultera i ännu högre prestanda förbättringar.
  \end{sweabstract}
\end{otherlanguage}